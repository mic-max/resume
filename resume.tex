\documentclass[11pt, a4paper]{micmax}

\geometry{left=1.4cm, top=.8cm, right=1.4cm, bottom=1.8cm, footskip=.5cm}
\fontdir[fonts/]
\definecolor{awesome}{HTML}{0387c9}

% Colors for text
% Uncomment if you would like to specify your own color
% \definecolor{darktext}{HTML}{414141}
% \definecolor{text}{HTML}{333333}
% \definecolor{graytext}{HTML}{5D5D5D}
% \definecolor{lighttext}{HTML}{999999}

% Set false if you don't want to highlight section with awesome color
\setbool{acvSectionColorHighlight}{true}

% If you would like to change the social information separator from a pipe (|) to something else
\renewcommand{\acvHeaderSocialSep}{\quad\textbar\quad}


%-------------------------------------------------------------------------------
%	PERSONAL INFORMATION
%	Comment any of the lines below if they are not required
%-------------------------------------------------------------------------------
% Available options: circle|rectangle,edge/noedge,left/right
% \photo[rectangle,edge,right]{./examples/profile}
\name{Michael}{Maxwell}
\position{Software Engineer{\enskip\cdotp\enskip} Ottawa, Canada}
% \address{Ottawa, Canada}

\mobile{+1 613 795 4472}
\email{yo@micmax.pw}
\homepage{www.micmax.pw}
\github{mic-max}
\linkedin{micmax}

%-------------------------------------------------------------------------------
\begin{document}

% Print the header with above personal informations
% Give optional argument to change alignment(C: center, L: left, R: right)
\makecvheader[C]

% Print the footer with 3 arguments(<left>, <center>, <right>)
% Leave any of these blank if they are not needed
\makecvfooter
  {\today}
  {Michael Maxwell~~~·~~~Résumé}
  {\thepage}


%-------------------------------------------------------------------------------
%	CV/RESUME CONTENT
%	Each section is imported separately, open each file in turn to modify content
%-------------------------------------------------------------------------------
\cvsection{Summary}

\begin{cvparagraph}

Current Senior at Carleton with an expected graduation date in April 2019. 5+ years of programming experience, with a focus in backend development and software engineering. In between schoolwork I like attending hackathons and exploring hobbyist electronics. I'm very interested in parallel \& distributed computing where performance metrics are critical. Seeking job opportunities in North America for a role I can mature into and become an effective addition to the team.

\end{cvparagraph}
\cvsection{Skills}

\begin{cvskills}

  \cvskill
    {Languages} % Category
    {C, C++, Go, Java, Node.js, JavaScript, SQL, HTML, CSS, \LaTeX} % Skills

  \cvskill
    {Frameworks / Libraries} % Category
    {JUnit, Cucumber, Mocha, Vue.js, Express, MongoDB, jQuery, socket.io, Cilk++, MPI} % Skills

  \cvskill
    {Development} % Category
    {Git, Eclipse, Jira, Agile, SCRUM, Linux, AWS, Docker, Test-driven development} % Skills

\end{cvskills}
\cvsection{Work Experience}

\begin{cventries}
	\cventry
		{Software Engineer --- Edge Experimentation}
		{Microsoft}
		{Redmond, Washington}
		{Jun. 2019 - Present}
		{\begin{cvitems}
			% Feature Flags in Code Parsing
			\item Parsing feature flags information from source code such as default states per platform to create useful systems for our engineers to understand the current state of the thousands of flags and monitor changes we receive from Chromium.
			% \item static code analysis
			% Chrome Experimentation Data Ingestion
			% Approval Bot
			\item Developing rules for a bot that is part of a pull request validation workflow, knowledgable on Azure functions and how to use them with ADO web hooks and other serverless resources available on Azure.
		\end{cvitems}}

	\cventry
		{Software Developer --- Summer Student}
		{Martello Technologies}
		{Kanata, Canada}
		{Apr. 2018 - Aug. 2018}
		{\begin{cvitems}
			\item Replaced a legacy dashboard with a Vue.js application which iteracted with our REST API
			\item Developed resuable custom Vue.js UI components for our internal library (e.g. a modal, a filtered sorted table)
			\item Created a suite of Mocha tests on currently supported CRUD operations of containers and devices
			\item Presented my progress at biweekly sprint meetings to the management team, shareholders, and all other developers
		\end{cvitems}}

	\cventry
		{Teaching Assistant}
		{Carleton University: School of Computer Science}
		{Ottawa, Canada}
		{Sept. 2016 - Dec. 2018}
		{\begin{cvitems}
			\item Comp 2406: Fundamentals of Web Applications \& Comp 1406: Introduction to Computer Science II
			\item Held office hours, organised and ran weekly workshops following the course's content and assignments
			\item Graded assignments using a rubric that all teaching assistants collaborated in creating together
		\end{cvitems}}
\end{cventries}
\cvsection{Projects}

\begin{cventries}

  \cventry
    {Co-Lead Developer --- Created for a 3rd year software engineering course} % Affiliation/role
    {Android Application - Blitz SMS} % Organization/group
    {Ottawa, Canada} % Location
    {Sept. 2017 - Jan. 2018} % Date(s)
    {
      \begin{cvitems} % Description(s) of experience/contributions/knowledge
        \item An android application that can make requests to our multiple services and presents the data in a relevant manner
        \item Server makes API requests to answer the user's request and send that back using the linked twilio number
        \item Managed a team of four students and assigned weekly tasks, reviewed pull requests, and developed the server using a unique transfer protocol
        \item Our client-server architecture made use of several design patterns such as: Singleton, Strategy, Template Method, Facade, and Retry
      \end{cvitems}
    }

  \cventry
    {Design inspired by a Ben Eater series on YouTube} % Affiliation/role
    {8 Bit Breadboard CPU} % Organization/group
    {Ottawa, Canada} % Location
    {Sep. 2017 - Oct. 2018} % Date(s)
    {
      \begin{cvitems} % Description(s) of experience/contributions/knowledge
        \item Recreated an 8 Bit CPU on a series of breadboards using mostly 74LS TTL components to develop a greater understanding of the fundamentals of how computers function. I was able to create and run very simple programs in the Turing complete system's byte code
        \item Goal: Scale it to 16 bits and add more ram for more complex programs and create a simple compiler for its bytecode
      \end{cvitems}
    }

\end{cventries}

\cvsection{Honors \& Awards}

\begin{cvhonors}

  \cvhonor
    {NSERC Experience Award} % Award
    {Martello Technologies} % Event
    {Kanata, Canada} % Location
    {2018} % Date(s)

  \cvhonor
    {Diplôme d’Études en Langue Française} % Award
    {DELF B1} % Event
    {Ottawa, Canada} % Location
    {2015} % Date(s)

  % \cvhonor
  %   {Double Blue Society} % Award
  %   {Osgoode Township High School} % Event
  %   {Metcalfe, Canada} % Location
  %   {2011 - 13} % Date(s)

\end{cvhonors}

\cvsection{Education}

\begin{cventries}

  \cventry
    {Bachelor of Computer Science Honours: Software Engineering} % Degree
    {Carleton University} % Institution
    {Ottawa, Canada} % Location
    {Sept. 2015 - Apr. 2019} % Date(s)
    {
      \begin{cvitems} % Description(s) bullet points
        \item Relevant Coursework: 
        \begin{itemize}
          \item Parallel Programming for Clusters \& Multi-Core Processors, Design \& Analysis of Algorithms
          \item Software Quality Assurance, Software Product Management, Object-Oriented Software Engineering
        \end{itemize}
        \item David A. Thomas Scholarship in Computer Science for attaining a 10.0 CGPA
        \item Developer for cuHacking --- the student organized hackathon
      \end{cvitems}
    }

\end{cventries}

\end{document}